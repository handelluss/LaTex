\documentclass[a4paper, 14pt]{article}
\usepackage{import}
\import{C:/LaTex}{pream.tex}

\title{Дискретная математика. I Семестр}
\author{Лектор: Пузынина Светлана Александровна \\
        Автор конспекта: Буглеев Антон}
\date{2022}

\begin{document}    
    \maketitle
    \newpage

    \section{Булевы Функции}
    
    \subsection*{Булевы Функции. Базис}
    \begin{definition}
        {\it Булевой функцией} называется функция вида
        \begin{align*}
            f : \{0, 1\}^n \rightarrow \{0, 1\}.
        \end{align*}
    \end{definition}

    \begin{definition}
        {\it Базис} - некоторое множество булевых функций.
    \end{definition}
    
    \begin{definition}
        {\it Формула над базисом} определяется по индукции: \\
        {\it База}: всякая функция $f \in F$ является формулой над $F$ \\
        {\it Индуктивный переход}: если $f(x_1, ..., x_n)$ - формула над $F$,
        а $\Phi_1, ..., \Phi_n$ - переменные, либо формулы над $F$, то тогда
        $f(\Phi_1, ... \Phi_n)$ - тоже формула над $F$.
    \end{definition}


    \subsection*{ПК, ДНФ, СДНФ, ПД, КНФ, СКНФ, Многочлен (полином) Жегалкина}
    \begin{definition}
        {\it Простой конъюнкцией} (ПК) называется конъюнкция одной или нескольких переменных
        или их отрицаний, причём каждая переменная встречается не более одного раза.
    \end{definition}
    \begin{definition}
        {\it Дизъюнктивная нормальная форма} (ДНФ) - дизъюнкция простых конъюнкций
    \end{definition}
    \begin{definition}
        {\it Совершенная дизъюнктивная нормальная форма} (СДНФ) - ДНФ, в которой в
        каждой конъюнкции учавствуют все переменные. 
    \end{definition}

    Аналогично определяются {\itПростая дизъюнкция} (ПД), {\itКонъюнктивная нормальная форма} (КНФ),
    {\itСовершенная конъюнктивная нормальная форма} (СКНФ).

    \begin{definition}
        {\it Многочлен (полином) Жегалкина} - сумма по модулю 2 конъюнкций переменных
        без повторений слагаемых, а также (необязательно) слагаемое 1.
        \begin{align*}
            f(x_1, ..., x_n) = a \oplus a_1 \wedge x_1 \oplus ... \oplus 
            a_{12} \wedge x_1 \wedge x_2 
            \oplus ... \oplus a_{1..n} \wedge x_1 \wedge ... \wedge x_n
        \end{align*}
    \end{definition}

    Например, $f(x, y, z) = x \oplus x \wedge y \wedge z \oplus 1$ 

    \begin{theorem}
        Для каждой функции существует единственное представление многочленом Жегалкина.
    \end{theorem}
    \begin{proof}
        \dots
    \end{proof}
    
    \subsection*{Замыкание. Замкнутые классы. Полнота}
    
    \begin{definition}
        {\itЗамыканием $[F]$} базиса $F$ называется множество всех
        функций, представимых формулой над $F$
    \end{definition}
    \begin{definition}
        {\itЗамкнутый класс} - класс, равный своему замыканию: $F = [F]$
    \end{definition}

    \begin{enumerate}
        \item $T_0 = \{f \ \vert \ f(0, \dots, 0) = 0\}$
        \item $T_1 = \{f \ \vert \ f(1, \dots, 1) = 1\}$
        \item $S = \{f \ \vert \ f(x_1, \dots, x_n) = \lnot f(\lnot x_1, \dots, \lnot x_n)\}$
        \item $M = \{f \ \vert \ \forall \text{ двоичных наборов } \alpha \leq \beta: \
        f(\alpha) \leq f(\beta)\}$
        \item $L = \{f \ \vert \ f(x_1, \dots, x_n) = x_1 \oplus ... \oplus x_n \oplus c\}$, где $c \in \{0, 1\}$
    \end{enumerate}
    
    \begin{theorem}
        Классы $T_0, T_1, S, M, L$ являются замкнутыми.
    \end{theorem}
    \begin{proof}
        \dots
    \end{proof}

    \begin{definition}
        Множество булевых функций $F$ называется {\itполной системой},
        если все булевы функции выразимы как формулы над данным базисом.
    \end{definition}

    \begin{theorem}
        Множество булевых функций $F$ является полным тогда и только
        тогда, когда $F$ не содержится ни в одном из пяти классов
        $T_0, T_1, S, M, L$
    \end{theorem}
    \begin{proof}
        \begin{enumerate}
            \item $\RA$ \\
            \dots
            \item $\LA$ \\
            \dots
        \end{enumerate}
    \end{proof}

    \section{Комбинаторика}
    \subsection*{Выборки}
    \begin{definition}
        Введём $A = \{a_1, \dots, a_n\}$. Некоторый набор элементов
        $(a_{i_1}, \dots, a_{i_r})$ называется {\itвыборкой объёма $r$ из $n$ элементов} или {\it$(n,r)$-выборкой}.
    \end{definition}

    Выборки бывают {\it упорядоченные} (порядок элементов важен) или {\it неупорядоченные} (без разницы, в каком порядке элементы),
    а также {\it с повторениями} и {\it без повторений}.

    Пусть объект $A$ можно выбрать $n$ способами, а объект
    $B$ - $m$ способами. Тогда важны два правила:
    \begin{enumerate}
        \item {\it Правило суммы.} Выбор <<$A$ или $B$>> можно выбрать $n+m$ способами.
        \item {\it Правило произведения}. Выбор пары $(A, B)$ можно выбрать $nm$ способами.
    \end{enumerate}

    \begin{definition}
        Выборки $k$ элементов из $n$:
        \begin{enumerate}
            \item {\it Упорядоченная с повторениями}: $n^k$
            \item {\it Упорядоченная без повторений (размещения)}: $A^k_n = \dfrac{n!}{(n-k)!}$
            \item {\it Неупорядоченная без повторений (сочетания)}: $C^k_n = \dfrac{n!}{k!(n-k)!}$
            \item {\it Неупорядоченная с повторениями}: $\stackrel{{\wedge}}{C} = C^k_{n+k-1}$
        \end{enumerate} 
        \begin{proof}
            Пусть $A = \{a_1, \dots, a_n\}$. Неупорядоченная выборка $k$ элементов
            с повторениями задаётся вектором $(x_1, \dots, x_n)$, где $x_i$ - число повторений
            элемента $a_i$. Таким образом, $x_1 + \dots + x_n = k$

            Закодируем решение бинарным вектором $\underbrace{11 \dots 1}_{x_1} 0 \underbrace{11 \dots 1}_{x_2} 0 \dots 0\underbrace{11 \dots 1}_{x_n}$.
            Получаем вектор, состоящий из $k$ единиц и $(n-1)$ нулей. Число таких векторов: $C^k_{n-1+k}$, что и требовалось
        \end{proof}
    \end{definition}

    \subsection*{Полезные свойства сочетаний}

    \begin{theorem}
        $C^k_n = C^k_{n-1} + C^{k-1}_{n-1}$
    \end{theorem}
    \begin{proof}
        \begin{align*}
            &C^k_{n-1} + C^{k-1}_{n-1} = \\
            &\dfrac{(n-1)!}{k!(n-1-k)!} + \dfrac{(n-1)!}{(k-1)!(n-k)!} = \\
            &\dfrac{(n-k)(n-1)! + k(n-1)!}{k!(n-k)!} = \\
            &\dfrac{(n-1)!((n-k) + k)}{k!(n-k)!} = \\
            &\dfrac{n!}{k!(n-k)!} = C^k_n
        \end{align*}
    \end{proof}

    {\it Треугольник Паскаля} \dots

    \begin{theorem}
        Бином Ньютона. \[(a+b)^n = \sum_{k=0}^n C_n^k a^kb^{n-k}\]
    \end{theorem}
    \begin{proof}
        Член $a^kb^{n-k}$ участвует в разложение $(a+b)^n$ столько раз,
        сколько есть способов выбрать $a$ в $k$ множителях из $n$ - 
        а это $C^k_n$.
    \end{proof}

    \begin{lemma}
        Грубые оценки для $n!$:
        \[(n/e)^n < n! < n^n\]
    \end{lemma}
    \begin{proof}
        Верхняя оценка очевидна. Докажем нижнюю по индукции:
        \begin{enumerate}
            \item База: $(1/e)^1 < 1 \LRA 1/e < 1$
            \item Переход: пусть верно для $n$:
            \begin{align*}
                &n! > \left(\dfrac{n}{e}\right)^n \LRA \\
                &(n+1)n! > (n+1)\left(\dfrac{n}{e}\right)^n \\
                &(n+1)! > (n+1)\left(\dfrac{n}{e}\right)^n
            \end{align*}
            Теперь покажем, что
            \begin{align*}
                &(n+1)\left(\dfrac{n}{e}\right)^n > \left(\frac{n+1}{e}\right)^{n+1} \LRA \\  
                &e(n+1)n^n > (n+1)^{n+1} \LRA \\
                &en^n > (n+1)^n \text{ (верно в курсе матанализа)}
            \end{align*}   
        \end{enumerate}
    \end{proof}

    \begin{theorem}
        Формула Стирлинга. 
        \begin{align*}
            &n! = (1 + o(1))\sqrt{2\pi n}\left(\frac{n}{e}\right)^n \LRA \\
            &\frac{n!}{1+o(1)} = \sqrt{2\pi n}\left(\frac{n}{e}\right)^n
        \end{align*}
    \end{theorem}
\end{document}
