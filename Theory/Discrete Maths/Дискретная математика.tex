\documentclass[a4paper, 14pt]{article}
\usepackage{import}
\import{C:/LaTex}{pream.tex}

\title{Дискретная математика. I Семестр}
\author{Буглеев Антон}
\date{2022}

\begin{document}    
    \maketitle
    \newpage

    \section{Булевы Функции}
    
    \subsection{Булевы Функции. Базис}
    \begin{definition}
        {\bf Булевой функцией} называется функция вида
        \begin{align*}
            f : \{0, 1\}^n \rightarrow \{0, 1\}.
        \end{align*}
    \end{definition}

    \begin{definition}
        {\bf Базис} - некоторое множество булевых функций.
    \end{definition}
    
    \begin{definition}
        {\bf Формула над базисом} определяется по индукции: \\
        {\it База}: всякая функция $f \in F$ является формулой над $F$ \\
        {\it Индуктивный переход}: если $f(x_1, ..., x_n)$ - формула над $F$,
        а $\Phi_1, ..., \Phi_n$ - переменные, либо формулы над $F$, то тогда
        $f(\Phi_1, ... \Phi_n)$ - тоже формула над $F$.
    \end{definition}


    \subsection{ПК, ДНФ, СДНФ, ПД, КНФ, СКНФ, Многочлен (полином) Жегалкина}
    \begin{definition}
        {\bf Простой конъюнкцией} (ПК) называется конъюнкция одной или нескольких переменных
        или их отрицаний, причём каждая переменная встречается не более одного раза.
    \end{definition}
    \begin{definition}
        {\bf Дизъюнктивная нормальная форма} (ДНФ) - дизъюнкция простых конъюнкций
    \end{definition}
    \begin{definition}
        {\bf Совершенная дизъюнктивная нормальная форма} (СДНФ) - ДНФ, в которой в
        каждой конъюнкции учавствуют все переменные. 
    \end{definition}

    Аналогично определяются {\itПростая дизъюнкция} (ПД), {\itКонъюнктивная нормальная форма} (КНФ),
    {\itСовершенная конъюнктивная нормальная форма} (СКНФ).

    \begin{definition}
        {\bf Многочлен (полином) Жегалкина} - сумма по модулю 2 конъюнкций переменных
        без повторений слагаемых, а также (необязательно) слагаемое 1.
        \begin{align*}
            f(x_1, ..., x_n) = a \oplus a_1 \wedge x_1 \oplus ... \oplus 
            a_{12} \wedge x_1 \wedge x_2 
            \oplus ... \oplus a_{1..n} \wedge x_1 \wedge ... \wedge x_n
        \end{align*}
    \end{definition}

    Например, $f(x, y, z) = x \oplus x \wedge y \wedge z \oplus 1$ 

    \begin{theorem}
        Для каждой функции существует единственное представление многочленом Жегалкина.
    \end{theorem}
    \begin{proof}[Proof]
        \dots
    \end{proof}
    
    \subsection{Замыкание. Замкнутые классы. Полнота}
    
    \begin{definition}
        {\bfЗамыканием $[F]$} базиса $F$ называется множество всех
        функций, представимых формулой над $F$
    \end{definition}
    \begin{definition}
        {\bfЗамкнутый класс} - класс, равный своему замыканию: $F = [F]$
    \end{definition}

    \begin{enumerate}
        \item $T_0 = \{f : f(0, \dots, 0) = 0\}$
        \item $T_1 = \{f : f(1, \dots, 1) = 1\}$
        \item $S = \{f : f(x_1, \dots, x_n) = \lnot f(\lnot x_1, \dots, \lnot x_n)\}$
        \item $M = \{f : \forall \text{ двоичных наборов } \alpha \leq \beta
        f(\alpha) \leq f(\beta)\}$
        \item $L = \{f: f(x_1, \dots, x_n) = x_1 \oplus ... \oplus x_n \oplus c\}$, где $c \in \{0, 1\}$
    \end{enumerate}
    
    \begin{theorem}
        Классы $T_0, T_1, S, M, L$ являются замкнутыми.
    \end{theorem}
    \begin{proof}[Proof]
        
    \end{proof}

    \begin{definition}
        Множество булевых функций $F$ называется {\bfполной системой},
        если все булевы функции выразимы как формулы над данным базисом.
    \end{definition}

    \begin{theorem}
        Множество булевых функций $F$ является полным тогда и только
        тогда, когда $F$ не содержится ни в одном из пяти классов
        $T_0, T_1, S, M, L$
    \end{theorem}
    \begin{proof}
        \begin{enumerate}
            \item $\RA$ \\
            \dots
            \item $\LA$ \\
            \dots
        \end{enumerate}
    \end{proof}

    \section{Комбинаторика}

\end{document}