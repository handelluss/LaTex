\documentclass[a4paper, 14pt]{article}
\usepackage{import}
\import{C:/LaTex}{pream.tex}

\title{Основы теории множеств. I Семестр}
\author{Лектор: Селиванов Виктор Львович \\
        Автор конспекта: Буглеев Антон}
\date{2022}
\begin{document}
    \maketitle
    \newpage

    \section{Мощность. Характеристическая функция}

    \begin{definition}
        {\bfМощностью} $\abs{A}$ называется число элементов в $A$.
    \end{definition}

    \begin{definition}
        Фиксируем произвольное множество U, элементами которого являются множества
        $A_1, ..., A_n$. \\
        {\bfХарактеристической функцией (индикатором)} множества $X \subset U$ называют
        функцию $\chi_X(u)$ = 
        $\begin{cases} 
            1, u \in X \\
            0, u \notin X
        \end{cases}$
    \end{definition}

    Основные свойства, если $A, B \subset U$:
    \begin{enumerate}
        \item $\chi_{A \cap B} = \chi_A \cdot \chi_B$
        \item $\chi_{A\cup B} = \chi_A + \chi_B - \chi_{A \cap B}$
        \item $\chi_{A\triangle B} = \chi_A + \chi_B - 2\chi_{A \cap B}$
        \item $\chi_{A^c} = \chi_A$
    \end{enumerate}

    \begin{theorem} $\abs{A_1 \cup ... \cup A_n}$ равно
        \[
        \sum_{i} \abs{A_i} - 
        \sum_{i<j} \abs{A_i \cap A_j} + 
        \sum_{i<j<k} \abs{A_i \cap A_j \cap A_k} - ...
        \]
    \end{theorem}
    \begin{proof}[Proof]
        \[ \chi_{A_1 \cup ... \cup A_n} = 1 - (1 - \chi_{A_1}) \cdot ... \cdot (1 - \chi_{A_n}) \]
        Раскрыв скобки получаем

        \[
            \sum_{i} \chi_{A_i} -
            \sum_{i < j} \chi_{A_i}\chi_{A_j} +
            \sum_{i < j < k} \chi_{A_i}\chi_{A_j}\chi_{A_k}  - \dots  
        \]
        \[ 
        \sum_{i} \abs{A_i} -
        \sum_{i<j}  \abs{A_i \cap A_j} +
        \sum_{i<j<k} \abs{A_i \cap A_j \cap A_K} - \dots
        \] что и требовалось.
    \end{proof}

    \begin{theorem}
        $  \abs{A_1 \triangle ... \triangle A_n} $ равно
        \[
        \sum_i  \abs{A_i} -
        2\sum_{i<j} \abs{A_iA_j} +
        4\sum_{i<j<k} \abs{A_iA_jA_k} - \dots
        \]
    \end{theorem}
    
    \begin{definition}
        Множества называются {\bfРавномощными}, если между ними можно
        установить взаимно-однозначное соответствие.
    \end{definition}

    \section{Отношения}
    \begin{definition}
        {\bf Отношением} называется любое множество 
        $R \subset A \times B$, где $A$ и $B$ - какие-то множества
    \end{definition}
    \begin{definition}
        {\bf Бинарное отношение} - отношение вида $R \subset X \times X$
    \end{definition}

    Свойства отношений:
    \begin{enumerate}
        \item $\forall x \in X: xRx$ (рефлексивность)
        \item $\forall x,y \in X: xRy \RA  yRx$ (симметричность)
        \item $\forall x,y,z \in X: xRy \land yRz \RA xRz$ (транзитивность)
        \item $\forall x,y \in X: xRy \land yRx \RA a = b$ (антисимметричность)
        \item $\forall x,y \in X: xRy \lor yRx$ (связность)
    \end{enumerate}

    \begin{definition}
        {\bfОтношение эквивалентности} - всякое симметричное, рефлексивное и транзитивное отношение
    \end{definition}
    Пример: $X$ - множество прямых в плоскости, тогда всякие прямые $a, b \in X$
    находятся в отношении эквивалентности ($a \sim b$).\\

    {\it Отношение равномощности} есть отношение эквивалентности. Примеры: 
    \begin{enumerate}
        \item Множество бесконечных последовательностей единиц и нулей равномощно 
        множеству всех подмножеств натуральных чисел. ($(010101\dots)$ соответствует
        ряду чётных чисел)
        \item Множество подмножеств любого множества $U = P(U)$ равномощно
        множеству всех функций, которые ставят в соответствие каждому элементу
        $x \in U$ либо 0, либо 1. Другими словами, каждому $(X \subset P(U))$
        соответствует своя характеристическая функция
    \end{enumerate}
    \begin{definition}
        {\bfОтношение частичного порядка} - всякое рефлексивное, транзитивное и
        антисимметричное отношение.
    \end{definition}
    Пример: пусть $X = \mathcal{P}(M)$ - множество всех подмножеств множества $M$.
    Два произвольные множества $A, B \subset X$ находятся в отношении частичного
    порядка ($A \preceq B$).

    \begin{definition}
        {\bfОтношение линейного порядка} - всякое связное отношение частичного порядка.
    \end{definition}
    
    \section{Счётные множества}

    \begin{definition}
        Множество называется {\it счётным}, если оно равномощно $\N$.
    \end{definition}
    Пример: множество $\Z$ счётно, так как множество $\Z$ можно
    представить в виде: $\{0, 1, -1, 2, -2, \dots\}$.

    \begin{theorem}
        Подмножество счётного множества конечно или счётно.
    \end{theorem}
    \begin{proof}
        Возьмём счётное множество $A = \{a_0, a_1, a_2, \dots\}$.
        Множество $B$ образуем следующим образом: вычёркиваем из
        $A$ те элементы, которые $\notin B$, сохраняя порядок
        оставших. Очевидно, оставшиеся члены образуют бесконечную
        последовательность (тогда $B$ - счётно, так как сохранился
        пронумерованный порядок),  либо $B$ конечно.
    \end{proof}
    
    \begin{theorem}
        Всякое бесконечное множество содержит счётное множество.
    \end{theorem}
    \begin{proof}
        Пусть имеется бесконечное множество $B$. Возьмём некоторый
        элемент $b_1 \in B$. Так как $B$ бесконечно, возьмём какой-либо
        другой элемент $b_2 \in B$, и так далее. Множество
        $\{b_1, b_2, \dots, b_n\} \subset B$ является счётным. 
    \end{proof}
    \begin{theorem}
        Объединение конечного или счётного числа конечных или
        счётных множеств конечно или счётно.
    \end{theorem}
\end{document}