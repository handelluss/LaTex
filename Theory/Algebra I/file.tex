\documentclass[a4paper, 14pt]{article}
\usepackage{import}
\import{C:/LaTex}{pream.tex}

\title{Алгебра. I Семестр}
\author{Лектор: Вавилов Николай Александрович \\
        Автор конспекта: Буглеев Антон}
\date{2022}

\begin{document}
    \maketitle
    \newpage

    \section{Некоторые бинарные операции}
    \subsection*{Операции над векторами}
    Сложение и умножение векторов: 
    \[(x_1, ..., x_n) + (y_1, ..., y_n) = (x_1 + y_1, ..., x_n + y_n)\]
    \[(x_1, ..., x_n)(y_1, ..., y_n) = (x_1y_1, ..., x_ny_n)\]

    Комплексное умножение:
    \[(a,b)(c,d) = (ac - bd, ad+bc)\]

    Векторное умножение в $\mathbb{R}^3$:    
    \[ (x_1,x_2,x_3) \times (y_1,y_2,y_3) = (x_2y_3 - x_3y_2, -x_1y_3 + x_3y_1, x_1y_2 - x_2y_2)\]
    

    \subsection*{Операции над матрицами}
    Сложение матриц:
    \[
    \begin{pmatrix}
        a & b \\
        c & d
    \end{pmatrix}
    +
    \begin{pmatrix}
        e & b \\
        g & h
    \end{pmatrix}
    =
    \begin{pmatrix}
        a+c & b+f \\
        c+g & d + h
    \end{pmatrix}
    \]

    Умножение матриц:
    \[
        \begin{pmatrix}
            a & b \\
            c & d    
        \end{pmatrix}
        \cdot
        \begin{pmatrix}
            e & f \\
            g & h
        \end{pmatrix}
        =
        \begin{pmatrix}
            ae+bg && af+bh \\
            ce+dg && cf+ch
        \end{pmatrix}
    \]
    
    \section{Структуры}
    \subsection*{Основные структуры}
    $X \neq \emptyset \\
    *: X \times X \rightarrow X \\
    (x,y) \mapsto x * y$ \\

    Аксиомы:
    \begin{enumerate}
        \item $\forall x,y,z \in X: x * (y * z) = (x * y) * z$ (Ассоциативность)
        \item $\exists e \in X: e * x = x = x * e$ (нейтральный элемент)
        \item $\forall x \in X, \exists x': x * x' = x' * x = e$ (обратный элемент)
        \item $\forall x,y \in X: a * b = b * a$ (коммутативность)
    \end{enumerate}
    
    \begin{definition}
        {\bfПолугруппа (Semigroup)} - множество $X$ с операцией, удовлетворяющее аксиоме 1
    \end{definition}
    Примеры: $(\mathbb{N}, +)$
    \begin{definition}
        {\bfМоноид (Monoid)} - множество $X$ c операцией *, удовлетворяющее аксиомам 1-2
    \end{definition}
    Примеры: $(\mathbb{N}_0, +)$, $(\mathbb{N}, *)$, $(X, \cup)$
    
    \begin{definition}
        {\bfГруппа (Group)} - множество $X$ c операцией *, удовлетворяющее аксиомам 1-3
    \end{definition}

    \begin{definition}
        {\bfАбелева (коммутативная) группа (Abelian group)} - множество $X$ c операцией *, удовлетворяющее аксиомам 1-4
    \end{definition}

    \subsection*{Некоторые полезные леммы и определения}

    \begin{definition}
        Элемент $z \in X$ называется {\bfрегулярным}, если $\forall x, y \in X:$
        \[
        \begin{cases}
            x*z=y*z \RA x=y \text{ (Регулярный справа)}\\
            z*x=z*y \RA x=y \text{ (Регулярный слева)}
        \end{cases}
        \]
    \end{definition}

    \begin{definition}
        Элемент $z \in X$ называется {\bfобратимым}, если $\exists z' \in X:$
        \[
            \begin{cases}
                z * z' = e \text{ (Обратимый слева)} \\
                z' * z = e \text{ (Обратимый справа)}
            \end{cases}  
        \]
    \end{definition}

    \begin{lemma}
        Элемент $z \in X$ обратим слева/справа $\RA$ $z$ регулярен слева/справа
    \end{lemma}
    \begin{proof}[Proof]
        \dots
    \end{proof}

    \begin{lemma}
        В группе $G$ есть левое и правое деление: 
        \[\forall h,g \in G \ \existu x,y \in G, (hx = g) \land (yh = g) \RA (x = h^{-1}g) \land (y=gh^{-1})\]
    \end{lemma}
    \begin{proof}
        Докажем, что $hx=g \RA x=h^{-1}g$
        \begin{align*}
            hx &= g \ | \text{ домножим на $h^{-1}$}\\
            h^{-1}(hx) &= h^{-1}g \\
            (h^{-1}h)x &= h^{-1}g \\
            ex &= h^{-1}g \\
            x &= h^{-1}g
        \end{align*}
        Аналогичное доказательство утверждения $yh=g \RA y=gh^{-1}$
    \end{proof}
    
    \begin{definition}
        $H \subset G$ называется {\bfПодгруппой в $G$}, если 
        \[\forall x,y \in H, \ xy^{-1} \in H \LRA 
        \begin{cases}
            xy \in H \\
            y^{-1} \in H
        \end{cases}\]
    \end{definition}
    Примеры: 
    \begin{enumerate}
        \item $\R_{>0} < \R^{*}$ значит, что $\R_{>0}$ - подгруппа $\R^{*}$
        \item $\Q_{>0} < \Q^{*}$ значит, что $\R_{>0}$ - подгруппа $\R^{*}$
    \end{enumerate}


    \begin{definition}
        Операция {\bfвозведения в степень в моноиде.} Пусть $X$ - моноид с нейтральным $e$, $x \in X, \ n \in \N_0$. Тогда:
        \[x^0 = e, \ x^n = 
        \begin{cases}
            \left(x^{\frac{n}{2}}\right)^2, \ 2 \mid n \text{ ($2$ - делитель $n$)}\\
            x^{n-1} \cdot x, \ 2 \nmid n \text{ ($2$ -  не делитель $n$)}
        \end{cases}\]
    \end{definition}
    
    \begin{definition}
        Операция {\bfвозведения в степень в группе} определяется аналогично, только показатель $n \in \Z$
    \end{definition}
    
    \begin{definition}
        Группа $G$ называется {\bfконечной}, если её порядок $\abs{G}$ конечен
    \end{definition}
    
    \begin{definition}
        {\bfСимметрическая группа} множества $X$: \[S_X = \text{биекция }X \rightarrow X \]
    \end{definition}
    
    \section{Кольца и поля}

    \begin{definition}
        {\it Кольцом} называется множество $K$ с операцией сложения и умножения, обладающая
        следующими свойствами:
        \begin{enumerate}
            \item Относительно сложения существует абелева группа.
            \item Выполняется дистрибутивность: $\forall a, b, c \in K : a(b + c) = ab + ac \land (b+c)a = ba + ca$.   
        \end{enumerate}
    \end{definition}

    Следствия аксиом кольца:
    \begin{enumerate}
        \item $\forall a \in K : a0 = 0a = 0$
        \item $\forall a, b \in K : a(-b) = (-a)b = -ab$
        \item $\forall a, b, c \in K : a(b-c) = ab - ac$
    \end{enumerate}

    \begin{definition}
        Кольцо $K$ называется {\it коммутативным}, если выполнено $\forall a, b \in K : ab = ba$
    \end{definition}
    \begin{definition}
        Кольцо $K$ называется {\it ассоциативным}, если выполнено $\forall a, b, c \in K : (ab)c = a(bc)$
    \end{definition}
    \begin{definition}
        Кольцо с единицей называется {\it  унитальным}.
    \end{definition}
    Два важных замечания:
    \begin{enumerate}
        \item Если 1 = 0, то
        \[\forall a \in K : a = a1 = a0 = 0\]
        То есть кольцо $K$ состоит только из нуля. Если кольцо содержит более одного элемента, то $1 \neq 0$

        \item При наличии коммутативности умножения из двух тождеств дистрибутивности
        можно оставить только одно.
    \end{enumerate}

    Примеры колец:
    \begin{enumerate}
        \item $\Z, \Q, \R$ являются коммутативными ассоциативными кольцами с единицей
        относительно операций сложения и умножения.
        \item $2\Z$ является коммутативным ассоциативным кольцом без единицы.
        \item Множество векторов пространства с операциями сложения и векторного умножения
        является некоммутативным неассоциативным кольцом. Однако выполнены другие тождества:
        \begin{enumerate}
            \item $a \times b + b \times a = 0$ (антикоммутативность)
            \item $(a \times b) \times c + (b \times c) \times a + (c \times a) \times b = 0$ (тождество Якоби)
        \end{enumerate} 
    \end{enumerate}
    
    \begin{definition}
        Элемент $a'$ кольца с единицей называется {\it обратным} к элементу $a$,
        если выполнено $a'a = aa' = 1$.

        Элемент, имеющий обратный, называется {\it обратимым}
    \end{definition}

    \begin{definition}
        {\it Телом} называется унитальное ассоциативное кольцо, у которого $\forall x \in R\backslash{0} \ \exists x' : xx' = x'x = 1$    
    \end{definition}

    \begin{definition}
        {\it Полем} называется коммутативное ассоциативное кольцо с единицей,
        в котором каждый ненулевой элемент обратимым. 

        Примером полей являются $\R$ и $\Q$, но $\Z$ не является полем (т.к. обратимы только $\pm 1$)
    \end{definition}
    

    Важное свойство поля: \[ab = 0 \RA a = 0 \lor b = 0\]

    Кольцо $\Z$ также обладает этим свойством. Такие кольца называют 
    {\it кольцами без делителей нуля}.

    В кольце без делителей нуля имеет место:
    \[ac = bc \land c \neq 0 \RA a = b\]

    \begin{definition}
        Операция $\circ$ называется {\it самодистрибутивной (слева/справа)}, если
        \[x \circ (y \circ z) = (x \circ y) \circ (x \circ z)\]
        Например, конъюнкция и дизъюнкция являются {\it самодистрибутивными}
    \end{definition}
    
    

    Цитата Великого: {\it "К некоммутативности привыкаешь за семестр, к неассоциативности - никогда"}.


\end{document}